\section{The Safe Area}
In the multidimensional algorithms, non-faulty processes exchange messages
containing vectors. Each process in the system will exchange their vectors
in each round. And then they will compute one result just in the safe area.
The safe area is convex hull of the all the input vectors\cite{perles2007generalization}.

We can consider to use the linear algorithm to compute the safe area.
The goal is to find a vector $w$ that could be expressed as a convex conbination
of vectors in $C\prime$ for all choices $C\prime \in C$ such that $|C\prime| = n - f$. 
The linear program uses the $d + \binom{n}{n - f}(n - f)$ variables described below\cite{mendes2015multidimensional}:


-- $w_1, ..., w_{d}$: variables for $w_{i}$-th element of vector $w$, $1 <= i <= d$.

-- $\alpha_{C\prime, i}$: coefficients multiplying vectors of $C\prime$ that express $w$
as their convex combination. We include here only those $n - f$ indexes $i$ for which $v_{i} \in C\prime$.

For every $C\prime$, the linear constraints are as follows.

-- $w = \Sigma_{v_{i} \in C\prime}\alpha_{C\prime, i} \cdot v_{i}$ (i.e., $w$ is a linear combination of vectors in $C\prime$)

-- $\Sigma_{v_{i} \in C\prime}\alpha_{C\prime, i} = 1$(i.e., the sum of all coefficients for a particular $C\prime is 1$)

-- $\alpha_{C\prime, i} >= 0$ for all $v_{i} \in C\prime$ (i.e., all coefficients are nonnegative).

For all every $C\prime$, we get $d + 1 + n - f$ linear constraints, yielding a total of 
$\binom{n}{n - f}(d + 1 + n - f)$ constraints in $d + \binom{n}{n - f}(n - f)$ variables. 
Hence, for any fixed $f$, the vector $w$ can be found in ploynomial time by linear program
with the number of variables and constraints that are polynomial in n and d (but not in f).
However, when f grows with n, the computational compulexity is high. Observe that we are
interested in any feasible vector $w$ that satisfies the above linear constaints and any
deterministic optimization objectives function can be used in the linear program.
