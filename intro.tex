\section{Introduction}
This paper describe two approximate multidimension concensus algorithms in distributed system.
These algorithms are different from traditional concensus algorithms.  They are meant to resolve
the Byzantine problem which distributed system contains arbitrary failures. In traditional Byzantine
problem: there are n processes in the system, several of them are faulty processes which can be
considered generate any possible output in the system. Each non-faulty process will propose one
value and then they will get one concensus value which need to meet several conditions:

\begin{itemize}

    \item Termination: Every correct process eventually delivers some message

    \item Agreement: If a correct process delivers value m, then all correct processes deliver m

    \item Nontriviality: Both values should be possible outcomes. This property eliminates the
protocals that returns a fixed value independent of the initial input
\end{itemize}

In multidimensional system, all the processes will propose one vector of values, and 
all non-faulty processes will get concensus on the n-dimensional value.  The 
multidimensional input which is d-dimensional vector can be considered as a point in
d-dimensinal Euclidean space with $d > 0$. In the multidimentional Byzantine Concensus
problem, the out come of each process should also be identical. And the output value
need to be in the convex hull of the non-faulty processes' input in the d-dimensinal
Euclidean space.

To solve this problem, reasearchers also propose another problem named Byzantine
Approximate Agreement problem. This problem also defined the out come of non-faulty
processes will be in the convex hull. The outputs shold be within the Euclidean distance
$\epsilon$ of each other.

In these problem, the algorithms defined in a model include following property:

\begin{itemize}

    \item All message will be eventually delivered

    \item Any two processes is connected to each other

    \item The communication channel is FIFO channel

    \item The processes can identify the sender by the sender ID in the message
\end{itemize}

From Vaidya and Garg's obversation, simply performing scalar consensus on
each dimension of the input vectors independently does not solve the vector consensus problem.
In particular, even if validity condition for scalar consensus is satisfied for each dimension of the
vector separately, the above validity condition of vector consensus may not necessarily be satisfied.
For instance, suppose that there are four processes, with one faulty process\cite{vaidya2013byzantine}.

\subsection{Multidimensional Byzantine Concensus}

For synchronous system, the algorithms will run in round by round. In each round, processes
will send messages and receive messages which were sent in this round.

A protocal solving the Multidimensional Byzantine Concensus problem need to satisfy following
conditions\cite{mendes2015multidimensional}:

\begin{itemize}
    \item Agreement. The output vector at all the non-faulty processes must be identical.

    \item Validity. The output vector at all non-faulty processes must be in the convex hull of
the non-faulty inputs.

    \item Termination. Each non-faulty process must terminate within a finite amount of time.

\end{itemize}

This is known that $n > 3f$ is necessary and sufficient to solve the scalar consensus, under the condition
that the commnunication model is a complete graph. 


\subsection{Multidimensional Byzantine Approximate Agreement}
In asynchronous systems, the message deliver time is not guarenteed. The message may take unbound time
to deliver. Also, there is not disjoint round in the algorithmes. It is not possible to identify a
process is faulty or slow[attiya2004distributed]. And it is well-known that asynchronous scalar consensus
is impossible in the presence of even a single crash failure\cite{fischer1985impossibility}. Here we
discuss the algorithms are also under the same condition, but the input and output switched to a
vector values.

A protocol satisfying these conditions could be considered solving the Multidimensional Byzantine Approximate
Agreement problem:

\begin{itemize}
    \item Agreement. The output vectors of non-faulty processes should be within Euclidean distance $\epsilon > 0$, a constant defined a priori. 

    \item Validity. The output vector at all non-faulty processors must be inside the convex hull of the input inputs.

    \item Termination. Each non-faulty process must terminate within a finite amount of time.

\end{itemize}


