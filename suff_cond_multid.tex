\section{Sufficient Condition for Multidimensional Approximate Agreement}
This paper introduces two different algorithms which are meant to resolve
the multidimensional approximate agreement problem. Both of the algorithms
were obtained by suitably modifiying Abraham's algorithm for approximate
agreement over scalars, which called AAD algorithm\cite{abraham2004optimal}. 

\subsection{The AAD Algorithm}
In the AAD algorithm, each non-faulty process $p_{i}$ maintains a scalar
variable $v_{i}$ that changes between multiple discrete rounds. The scalar
value in process $p_{i}$ at the end of round $r$ is denoted by $v_{i}^{r}$.
The input value of process $p_{i}$ is denoted by $v_{i}^{0}$. In each round,
non-faulty processes\cite{mendes2015multidimensional}:

1. Reliable broadcast the current value $v_{i}^{r-1}$;

2. Using the witness technique , receiving $M$, a message set containing
values from existing proecesses;

3. Compute a new state $v_{i}^{r}$, based on Content($M$)

\subsection{The Mendes-Herlihy Algorithm}
Medes-Herlihy's algorithm will approximate agree over vectors,
originally present in \cite{mendes2013multidimensional}. From the
algorithm it seems that the algorithm compute dimension by dimension
but each dimension do not compute independently. The algorithm using 
another sub-procedure to compute the number of iterations. For each
dimension, indexed by m, it execute a specific time to get convergence.
The algorithm will converge after $accept > f$ halt messages,
accumulated in $H$.

The Mendes-Herlihy algorithm is:

\begin{algorithm}
\caption{$p$.AsyncAgreeMH($I$)}
\begin{algorithmic}

    \State $(R, v) \leftarrow CalculateRounds(I)$
    \For{i} {1}... {d}
        \State $H \leftarrow \phi$
        \State $r \leftarrow 1$

        \While {$|H| <= f$} 
        \State RBSend(($p, m.r, v$))
        \State upon $V \leftarrow RBReceiveWitness(m.r)$ do

        \State \hspace{0.8cm} $S \leftarrow Safe_{f}(V)$

        \State \hspace{0.65cm} $v \leftarrow v \in S$ such that $v[m] = Midpoint(S(m))$
        \hspace{1cm} \If {$r = R$}
        \hspace{1cm} \State RESend(($p, m.r, \{halt\}$))
        \hspace{1cm} \EndIf

        \hspace{0.8cm} $r \leftarrow r + 1$
        \State upon RBRecv(($p\prime, m.r\prime, \{halt\}$)), with $r\prime >= r$ do

        \hspace{1cm} $H \leftarrow H \cup \{(p\prime, m.r\prime, \{halt\})\}$
        \EndWhile
    \EndFor

    \State return $v$
\end{algorithmic}
\end{algorithm}


\subsection{VG}

The sub-procedure of computing iteration times algorithm:

\begin{algorithm}
\caption{$p$.CalculateRounds($I$)}
\begin{algorithmic}
    \State RBSend(($p, 0, I$))
    \State $(V, W) \leftarrow (Val, Content(Wit))$ from RBReceiveWitness(0)
    \State $U \leftarrow$ \{barycenter of $Safe_{f}({W\prime})$: $W\prime \in W$\}
    \State $v \leftarrow$ barycenter of $Safe_{f}(U)$
    \State $R \leftarrow \left\lceil \log_2(\sqrt{d}/\epsilon \cdot max\{\delta_{U}(m):1 <= m <= d\})  \right\rceil$
    \State return ($R, v$)
\end{algorithmic}
\end{algorithm}

\subsection{The Vaidya-Garg Algorithm}
This algorithm works just like AAD algorithm. And it use simpler gemetric primitives.
This algorithm is present in \cite{vaidya2013byzantine}. The algorithm is following\cite{mendes2015multidimensional}:

\begin{algorithm}
\caption{$p$.AsyncAgreeVG($I$)}
\begin{algorithmic}

    \State $R \leftarrow 1 + \left\lceil \log_{1/(1-\gamma)}\frac{\sqrt{d}(U - \nu)}{\epsilon} \right\rceil $
    \For {i} {1}... {$R$}
        \State RBSend(($p, r, v$))
        \State upon $M \leftarrow$ RBReceiveWitness($r$) do

        \hspace{0.2cm} for $M\prime \subseteq M, |M\prime| = n - f $ do

        \hspace{0.4cm} $S_{M\prime} \leftarrow Safe_{f}(M\prime)$

        \hspace{0.4cm} $Z \leftarrow Z \cup DeterministicallyChoosePoint(S_{M\prime})$

        \hspace{0.2cm} $v \leftarrow (\Sigma_{z \in Z}z)/|Z|$
    \EndFor

    \State return $v$
\end{algorithmic}
\end{algorithm}
