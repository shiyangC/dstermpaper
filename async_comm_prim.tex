\section{Asynchronous Communication Primitives}

The multidimensional algorithms use two important commnunication
primitives: the reliable broadcast and the witness technique.

\subsection{Reliable Broadcast}


The reliable broadcast technique avoids the situation where Byzantine
processes convey different contents to different processes in a single
round of commnunication.  And the original paper is from Srikanth and 
Toueg\cite{srikanth1987simulating} and Bracha\cite{bracha1987asynchronous}.

The message sent by processes contains sender's identification. So 
one message contains the sender ID, receiver ID and content. And the reilable
broadcast technique has the following properties:

\begin{itemize}

    \item Non-faulty integrity: If a non-fauty processes p never reliably broadcasts one specific
    message, no other non-faulty process will ever receive it

    \item Non-faulty liveness: If a non-faulty process p does reliably broadcasts one message $m$, 
    all other non-faulty processes eventually receive message m

    \item Global uniqueness: If two non-faulty processes reliably receive message $m$ and $m\prime$,
    the messages are equal, even when the sender p is Byzantine.
    
    \item For two non-faulty processes $p_1$ and $p_2$, if $p_1$ reliably receive $m$, $p_2$ also
    reliably receive $m$, even when the sender p is Byzantine.
\end{itemize}

The algorithms are following:

\begin{algorithm}
\caption{$p$.RBSend(($p,r,c$))}
send($p, r, c$) to all processes

\end{algorithm}

\begin{algorithm}
\caption{$p$.RBEcho()}
\begin{algorithmic}

    \State upon recv($q, r, c$) from q do
    \If{never sent ($p, qr\{echo\}$, .)}
        \State send($(p, qr\{echo\})$, $c$) to all processes
    \EndIf

    \\
    \State upon recv(., qr\{echo\}, c) from $>= n - f$ processes do
    \If{never sent ($p, qr\{ready\}$, .)}
        \State send($(p, qr\{ready\})$, $c$) to all processes
    \EndIf
    \\

    \State upon recv(., qr\{ready\}, c) from $>= f + 1$ processes do
    \If{never sent ($p, qr\{ready\}$, .)}
        \State send($(p, qr\{ready\})$, $c$) to all processes
    \EndIf
    
\end{algorithmic}
\end{algorithm}

\begin{algorithm}
\caption{$p$.RBRecv(($q,r,c$))}
    recv(., qr\{ready\}, c) from $n - f$ processes

    return $(q, r, c)$

\end{algorithm}

\subsection{Witness Technique}
The witness technique provide a method which can make two non-faulty processes
get suitable overlaps values. This method is originally introduced by Abraham\cite{abraham2004optimal}.
The method can make sure that non-faulty processes have $n - f$ common values, which is 
essential for our correctness and optimality arguments. This witness technique will only
wait for messages certain to be delivered.

The algorithms are shown:

\begin{algorithm}
\caption{$p$.RBReceiveWitness($r$)}
\begin{algorithmic}

    \State $Val, Rep, Wit \leftarrow \phi$
    \While {$|Val| < n - f$}
        \State upon RBRecv(($p_{x} , r, c_{x}$)) do

        \hspace{1cm}  $Val \leftarrow Val \cup\{(p_{x}, r, c_{x})\}$
    \EndWhile
    \State RBSend(($p, r, Val$))
    \While {$|Wit| < n - f$} 
        \State upon RBRecv(($p_{x} , r, c_{x}$)) do

        \hspace{1cm}  $Val \leftarrow Val \cup\{(p_{x}, r, c_{x})\}$
        \State upon RBRecv(($p_{x} , r, Val_{x}$)) do

        \hspace{1cm}  $Rep \leftarrow Rep \cup\{(p_{x}, r, Val_{x})\}$
        \State $Wit \leftarrow \{(p_{x}, r, Val_{x}) \in Rep: Val_{x} \subseteq Val\}$
    \EndWhile
    return $Val$
    
\end{algorithmic}
\end{algorithm}
